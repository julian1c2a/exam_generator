\documentclass[a4paper,11pt]{article}
\usepackage[utf8]{inputenc}
\usepackage[spanish]{babel}
\usepackage[top=2.5cm, bottom=2.5cm, left=2cm, right=2cm, headheight=2cm]{geometry}
\usepackage{tikz}
\usepackage{circuitikz}
\usepackage{tikz-timing}
\usepackage{amsmath}
\usepackage{amssymb}
\usepackage{array}
\usepackage{multirow}
\usepackage{colortbl}
\usepackage{enumitem}
\usepackage{float}
\usepackage{tcolorbox}
\usepackage{graphicx}
\usepackage{fancyhdr}
\usepackage{lastpage}
\usepackage{diagbox}
\usepackage{xcolor} % Para resaltar soluciones

\usetikzlibrary{calc}
\tcbset{colback=gray!5!white, colframe=gray!75!black, title=\textbf{ENUNCIADO}, fonttitle=\bfseries, boxrule=0.5mm, arc=2mm}

\newcolumntype{C}[1]{>{\centering\arraybackslash}p{#1}}
\newcolumntype{B}{>{\centering\arraybackslash}p{0.5cm}}

\pagestyle{fancy}
\fancyhf{}
\renewcommand{\headrulewidth}{0.4pt}
\renewcommand{\footrulewidth}{0.4pt}

\lhead{\includegraphics[height=1.5cm]{../../resources/png/image_38294.png}}
\chead{\textbf{UNIVERSIDAD DE MÁLAGA} \\ Departamento de Tecnología Electrónica}
\rhead{\textbf{Fundamentos de Electrónica} \\ 1er Cuatrimestre \\ Curso 2025/2026}

\lfoot{\small Examen de Teoría - Parte Digital}
%% \cfoot{\small Profesores: Almudena | Julián}
\rfoot{\small Página \thepage\ de \pageref{LastPage}}

\begin{document}

\begin{center}
    {\Large \textbf{Examen de Teoría - Parte Digital}} \\ \vspace{0.2cm}
    {\Large \textbf{21 de Enero de 2026}}
\end{center}

\vspace{0.5cm}
\noindent \textbf{Apellido y Nombre:} ............................................................................ \hfill \textbf{Grupo:} ....................
\vspace{0.5cm} \hrule \vspace{0.5cm}
\section*{Ejercicio 1: Sistemas de Representación (8 bits)}
\begin{tcolorbox}[title=Enunciado]
\noindent \textbf{a)} Complete la tabla. Registro de 8 bits. Si no es representable, escriba 'NR'.
\end{tcolorbox}

\textbf{Respuesta:}
\begin{table}[H] \centering \renewcommand{\arraystretch}{1.5}
\begin{tabular}{|c|c|C{2.8cm}|C{2.8cm}|C{2.8cm}|C{2.8cm}|} \hline
\rowcolor[gray]{0.9} \textbf{Id} & \textbf{Decimal} & \textbf{Binario Nat.} & \textbf{Compl. 2} & \textbf{Signo-Mag.} & \textbf{BCD} \\ \hline
a) &  &  & 11001001 &  &  \\ \hline
b) &  & 01000001 &  &  &  \\ \hline
c) &  &  & 11001100 &  &  \\ \hline
d) & -77 &  &  &  &  \\ \hline
\end{tabular} \end{table}
\begin{tcolorbox}[title=Enunciado (Parte b)]
\noindent \textbf{b)} Realice las siguientes operaciones aritméticas.
\end{tcolorbox}
\par \vspace{0.5cm} \noindent \textbf{1) Suma en Binario Natural:} Fila a + Fila b
\begin{center} \renewcommand{\arraystretch}{1.5}
\begin{tabular}{r|B|B|B|B|B|B|B|B|}
\tiny{Acarreo} &  &  &  &  &  &  &  &  \\ \cline{2-9}
Op. 1 &  &  &  &  &  &  &  &  \\ \cline{2-9}
Op. 2 &  &  &  &  &  &  &  &  \\ \hline \hline
\textbf{Res.} &  &  &  &  &  &  &  &  \\ \cline{2-9}
\end{tabular} \end{center}
\par \vspace{0.2cm}
\noindent \textit{¿Overflow? $\square$ \hspace{1cm} ¿Underflow? $\square$ \hspace{1cm} ¿Correcto? $\square$}
\par \vspace{0.3cm}
\noindent \hspace{0.5cm} \textbf{¿Por qué?}
\par \vspace{0.8cm}
\par \vspace{0.5cm} \noindent \textbf{2) Suma en Complemento a 2:} Fila a + Fila d
\begin{center} \renewcommand{\arraystretch}{1.5}
\begin{tabular}{r|B|B|B|B|B|B|B|B|}
\tiny{Acarreo} &  &  &  &  &  &  &  &  \\ \cline{2-9}
Op. 1 &  &  &  &  &  &  &  &  \\ \cline{2-9}
Op. 2 &  &  &  &  &  &  &  &  \\ \hline \hline
\textbf{Res.} &  &  &  &  &  &  &  &  \\ \cline{2-9}
\end{tabular} \end{center}
\par \vspace{0.2cm}
\noindent \textit{¿Overflow? $\square$ \hspace{1cm} ¿Underflow? $\square$ \hspace{1cm} ¿Correcto? $\square$}
\par \vspace{0.3cm}
\noindent \hspace{0.5cm} \textbf{¿Por qué?}
\par \vspace{0.8cm}

%%%%%%%%%%%%%%%%%%%%%%%%%%%%%%%%%%%%%%%%%%%%%%%%%%%%%%%%%%%%
% >>>>>> INICIO EJERCICIO 2: Diseño y Simplificación Lógica <<<<<<
%%%%%%%%%%%%%%%%%%%%%%%%%%%%%%%%%%%%%%%%%%%%%%%%%%%%%%%%%%%%
\newpage \section*{Ejercicio 2: Diseño y Simplificación Lógica}
\begin{tcolorbox}[title=Enunciado]
\noindent Dada la función definida por la siguiente tabla de verdad:
\begin{table}[H] \centering \renewcommand{\arraystretch}{1.2}
\begin{tabular}{|c|c|c|c|c|} \hline
\rowcolor[gray]{0.9} \textbf{A} & \textbf{B} & \textbf{C} & \textbf{D} & \textbf{F} \\ \hline
0 & 0 & 0 & 0 & \textbf{1} \\ \hline
0 & 0 & 0 & 1 & \textbf{1} \\ \hline
0 & 0 & 1 & 0 & \textbf{1} \\ \hline
0 & 0 & 1 & 1 & \textbf{1} \\ \hline
0 & 1 & 0 & 0 & \textbf{0} \\ \hline
0 & 1 & 0 & 1 & \textbf{1} \\ \hline
0 & 1 & 1 & 0 & \textbf{1} \\ \hline
0 & 1 & 1 & 1 & \textbf{1} \\ \hline
1 & 0 & 0 & 0 & \textbf{0} \\ \hline
1 & 0 & 0 & 1 & \textbf{0} \\ \hline
1 & 0 & 1 & 0 & \textbf{1} \\ \hline
1 & 0 & 1 & 1 & \textbf{0} \\ \hline
1 & 1 & 0 & 0 & \textbf{0} \\ \hline
1 & 1 & 0 & 1 & \textbf{0} \\ \hline
1 & 1 & 1 & 0 & \textbf{1} \\ \hline
1 & 1 & 1 & 1 & \textbf{0} \\ \hline
\end{tabular} \end{table}
\noindent Se pide:
\begin{enumerate}[label=\alph*)]
\item Obtener la expresión canónica (Minitérminos (Suma de Productos)).
\item Simplificar por Karnaugh.
\item Implementar con puertas \textbf{NAND}.
\end{enumerate} \end{tcolorbox}
\textbf{Espacio de Resolución:}
\input{components/ej2_kmap.tex}
\vspace{3cm}

%%%%%%%%%%%%%%%%%%%%%%%%%%%%%%%%%%%%%%%%%%%%%%%%%%%%%%%%%%%%
% >>>>>> INICIO EJERCICIO 3: Problema de Diseño Lógico <<<<<<
%%%%%%%%%%%%%%%%%%%%%%%%%%%%%%%%%%%%%%%%%%%%%%%%%%%%%%%%%%%%
\newpage \section*{Ejercicio 3: Problema de Diseño Lógico}
\begin{tcolorbox}[title=Enunciado]
\textbf{Contexto: Sistema de Seguridad de Bóveda Bancaria}
\begin{itemize}
\item A: Sensor Reloj (1=Laboral)
\item B: Llave Director (1=Si)
\item C: Llave Gerente (1=Si)
\item D: Código (1=OK)
\item Salida: Z: Apertura
\end{itemize}
\textit{Lógica: La puerta se abre siempre que el código de seguridad sea correcto, SALVO que estemos fuera de horario laboral y falte alguna de las llaves (Director o Gerente).}
\end{tcolorbox}
\textbf{1. Tabla de Verdad:}
\begin{table}[H] \centering \renewcommand{\arraystretch}{1.2}
\begin{tabular}{|c|c|c|c|c|} \hline
\rowcolor[gray]{0.9} \textbf{A} & \textbf{B} & \textbf{C} & \textbf{D} & \textbf{Z} \\ \hline
0 & 0 & 0 & 0 &   \\ \hline
0 & 0 & 0 & 1 &   \\ \hline
0 & 0 & 1 & 0 &   \\ \hline
0 & 0 & 1 & 1 &   \\ \hline
0 & 1 & 0 & 0 &   \\ \hline
0 & 1 & 0 & 1 &   \\ \hline
0 & 1 & 1 & 0 &   \\ \hline
0 & 1 & 1 & 1 &   \\ \hline
1 & 0 & 0 & 0 &   \\ \hline
1 & 0 & 0 & 1 &   \\ \hline
1 & 0 & 1 & 0 &   \\ \hline
1 & 0 & 1 & 1 &   \\ \hline
1 & 1 & 0 & 0 &   \\ \hline
1 & 1 & 0 & 1 &   \\ \hline
1 & 1 & 1 & 0 &   \\ \hline
1 & 1 & 1 & 1 &   \\ \hline
\end{tabular} \end{table}
\newpage \textbf{2. Mapa de Karnaugh:}
% --- COMPONENTE GENERADO: ej3_problem_kmap ---
% ID: ej3_problem_kmap
% Si quieres fijar este diseño, copia este archivo a 'resources/latex/ej3_problem_kmap.tex' y edítalo.
\begin{center}
\begin{table}[H] \centering \renewcommand{\arraystretch}{2}
\begin{tabular}{|c|c|c|c|c|c|c|} \hline
\multicolumn{3}{|c|}{\multirow{3}*{\Huge \textbf{Z}}} & \multicolumn{4}{c|}{\textbf{CD =}} \\ \cline{4-7}
\multicolumn{3}{|c|}{} & 00 & 01 & 10 & 11 \\ \cline{4-7}
\multicolumn{3}{|c|}{} & 00 & 01 & 11 & 10 \\ \hline
\multirow{4}*{\rotatebox{90}{\textbf{AB =}}} & 00 & 00 &  &  &  &  \\ \cline{2-7}
 & 01 & 01 &  &  &  &  \\ \cline{2-7}
 & 10 & 11 &  &  &  &  \\ \cline{2-7}
 & 11 & 10 &  &  &  &  \\ \cline{2-7}
\hline
\end{tabular} \end{table}
\end{center}

\vspace{1cm}
\noindent \textbf{3. Esquema Lógico:}
\vspace{4cm}

%%%%%%%%%%%%%%%%%%%%%%%%%%%%%%%%%%%%%%%%%%%%%%%%%%%%%%%%%%%%
% >>>>>> INICIO EJERCICIO 4: Análisis de Bloques MSI <<<<<<
%%%%%%%%%%%%%%%%%%%%%%%%%%%%%%%%%%%%%%%%%%%%%%%%%%%%%%%%%%%%
\newpage \section*{Ejercicio 4: Análisis de Bloques MSI}
\begin{tcolorbox}[title=Enunciado]
Dado el siguiente esquema lógico (SUMADOR):
% --- COMPONENTE GENERADO: ej4_msi_sumador ---
% ID: ej4_msi_sumador
% Si quieres fijar este diseño, copia este archivo a 'resources/latex/ej4_msi_sumador.tex' y edítalo.

\begin{center} \begin{tikzpicture}
    \draw[thick] (0,0) rectangle (4,4);
    \node at (2,2) {\textbf{SUMADOR 4 BITS}};
    
    % Entradas A y B (Buses)
    \draw[ultra thick] (-1.2, 3) -- (0,3); \node[left] at (-1.2, 3) {A}; 
    \node[above] at (-0.6, 3.1) {\scriptsize 4}; \draw[thick] (-0.7, 2.8) -- (-0.5, 3.2);
    
    \draw[ultra thick] (-1.2, 1) -- (0,1); \node[left] at (-1.2, 1) {B}; 
    \node[above] at (-0.6, 1.1) {\scriptsize 4}; \draw[thick] (-0.7, 0.8) -- (-0.5, 1.2);
    
    % Carry In (Parte izquierda abajo)
    \draw (-1, 0.5) -- (0, 0.5) node[midway, above]{Cin=0};
    
    % Salida S (Bus)
    \draw[ultra thick] (4,2.5) -- (5.2,2.5); \node[right] at (5.2, 2.5) {S};
    \node[above] at (4.6, 2.6) {\scriptsize 4}; \draw[thick] (4.5, 2.3) -- (4.7, 2.7);
    
    % Carry Out
    \draw (4,1.5) -- (5,1.5) node[right]{Cout};
\end{tikzpicture} \end{center}

\noindent Determine la salida S y el acarreo de salida Cout.
\end{tcolorbox} \vspace{5cm}

%%%%%%%%%%%%%%%%%%%%%%%%%%%%%%%%%%%%%%%%%%%%%%%%%%%%%%%%%%%%
% >>>>>> INICIO EJERCICIO 5: Sistemas Secuenciales <<<<<<
%%%%%%%%%%%%%%%%%%%%%%%%%%%%%%%%%%%%%%%%%%%%%%%%%%%%%%%%%%%%
\newpage \section*{Ejercicio 5: Sistemas Secuenciales}
\begin{tcolorbox}[title=Enunciado]
Síncrono (COUNTER) por Subida. FF T. Async \textbf{Set(asyn)} a nivel 0.
% --- COMPONENTE GENERADO: ej5_seq_circuit ---
% ID: ej5_seq_circuit
% Si quieres fijar este diseño, copia este archivo a 'resources/latex/ej5_seq_circuit.tex' y edítalo.
\begin{center} \begin{circuitikz}[scale=1.2, transform shape] \draw
(0,0) node[flipflop T, external pins width=0](FF1){Q0} (5,0) node[flipflop T, external pins width=0](FF2){Q1};
\draw (FF1.pin 2) ++(0,0) -- ++(-0.5,0) -- ++(0,-2.0) coordinate(clk_bus);
\draw (FF2.pin 2) ++(0,0) -- ++(-0.5,0) -- (clk_bus -| FF2.pin 2) -- (clk_bus);
\draw (clk_bus) -- ++(-1.0,0) node[left]{CLK};
\draw (FF1.pin 1) -- ++(-1,0) node[left]{E};
\draw (FF1.pin 6) -- (FF2.pin 1);
\draw (FF1.pin 6) -- ++(0,1.5) node[above]{Q0};
\draw (FF2.pin 6) -- ++(0,1.5) node[above]{Q1};
\draw (FF1.up) -- ++(0,0.5) coordinate(a);
\draw (FF2.up) -- ++(0,0.5) -- (a);
\draw (a) -- ++(0,0.5) node[above]{Set};
\end{circuitikz} \end{center}
\end{tcolorbox}
% --- PRUEBA DE ALINEACIÓN IZQUIERDA Y COMPRESIÓN VERTICAL ---
\begin{flushleft}
	% arraystretch=0 elimina el padding proporcional.
	\renewcommand{\arraystretch}{0}
	% extrarowheight negativo elimina altura fija extra. Ajusta este valor (-2pt, -5pt, etc.)
	\setlength{\extrarowheight}{0pt}
	
	\begin{tikztimingtable}[
		timing/slope=0,
		timing/xunit=0.60cm,
		timing/yunit=0.35cm,
		timing/coldist=2pt
	]
		CLK & 24{C} \\
		Set(asyn) & 2L 22H \\
		E & LLLLLLLLLHHHHLHLHHLLLLLH \\
		Q0 & [draw=none, fill=none] 24{} \\
		Q1 & [draw=none, fill=none] 24{} \\
		\extracode \tablegrid
	\end{tikztimingtable}
\end{flushleft}
\vspace{0.5cm}
\noindent \textbf{Se pide:}
\begin{enumerate}[label=\alph*)]
\item Completar el cronograma (salidas Q0, Q1).
\item Determinar la secuencia de estados.
\end{enumerate}
\end{document}