\documentclass[a4paper,11pt]{article}
\usepackage[utf8]{inputenc}
\usepackage[spanish]{babel}
\usepackage[top=2.5cm, bottom=2.5cm, left=2cm, right=2cm]{geometry}

% --- PAQUETES (Idénticos al entorno de producción) ---
\usepackage{tikz}
\usepackage{circuitikz}
\usepackage{tikz-timing}
\usepackage{amsmath}
\usepackage{amssymb}
\usepackage{array}
\usepackage{multirow}
\usepackage{colortbl}
\usepackage{enumitem}
\usepackage{float}
\usepackage{tcolorbox}
\usepackage{graphicx}
\usepackage{fancyhdr}
\usepackage{diagbox}
\usepackage{xcolor}

% --- CONFIGURACIÓN DE RUTAS ---
% Rutas de búsqueda para imágenes
% 1. ../../png/ -> Para encontrar logos en resources/png/ desde resources/latex/debugging/
% 2. ../../../ -> Para encontrar imágenes en la raíz (legacy)
\graphicspath{{../../png/}{../../../}}

% --- CONFIGURACIÓN DE LIBRERÍAS ---
\usetikzlibrary{calc}
\tcbset{colback=gray!5!white, colframe=gray!75!black, title=\textbf{ENUNCIADO}, fonttitle=\bfseries, boxrule=0.5mm, arc=2mm}

% --- DEFINICIONES PERSONALIZADAS ---
\newcolumntype{C}[1]{{>{\centering\arraybackslash}p{#1}}}
\newcolumntype{B}{{>{\centering\arraybackslash}p{0.5cm}}}

\begin{document}

\section*{Entorno de Depuración de Componentes}

% Instrucciones visuales
\begin{tcolorbox}[colback=blue!5!white, colframe=blue!75!black, title=Instrucciones]
    \begin{enumerate}
        \item Pega el código del componente que quieres arreglar en el archivo \textbf{\texttt{candidate.tex}} (en esta misma carpeta).
        \item Compila este documento (\texttt{test\_component.tex}).
        \item Edita \texttt{candidate.tex} hasta que se vea bien.
        \item Cuando termines, renombra o mueve \texttt{candidate.tex} a \texttt{../nombre\_final.tex}.
    \end{enumerate}
\end{tcolorbox}

\vspace{1cm}
\hrule
\vspace{0.5cm}

\begin{center}
    % --- AQUÍ SE INCRUSTA TU COMPONENTE ---
    \IfFileExists{candidate.tex}
    {
        % --- COMPONENTE GENERADO: ej5_seq_timing ---
% ID: ej5_seq_timing
% Si quieres fijar este diseño, copia este archivo a 'resources/latex/ej5_seq_timing.tex' y edítalo.
\begin{center}
	\resizebox{\textwidth}{!}{%
		\renewcommand{\arraystretch}{0}
		\begin{tikztimingtable}[timing/slope=0, x=2.0cm, y=0.35cm]
			CLK\hspace{1em} & 24{C} \\
			Set(asyn)\hspace{1em} & 2L 22H \\
			E\hspace{2.5em} & LLLLLLLLLHHHHLHLHHLLLLLH \\
			Q0 & [draw=none, fill=none] 24{} \\
			Q1 & [draw=none, fill=none] 24{} \\
			\extracode \tablegrid
		\end{tikztimingtable}%
	}
\end{center}
    }
    {
        \textcolor{red}{\textbf{Error: No se encuentra el archivo 'candidate.tex'.\\ Crea ese archivo y pega ahí tu código TikZ/LaTeX.}}
    }
    % --------------------------------------
\end{center}

\vspace{0.5cm}
\hrule

\end{document}
